\documentclass[twocolumn,twoside,10pt,a4paper]{article}

\usepackage[portuguese]{babel}  % portuguese
\usepackage{graphicx}           % images: .png or .pdf w/ pdflatex; .eps w/ latex

%% For iso-8859-1 (latin1), comment next line and uncomment the second line
\usepackage[utf8]{inputenc}
%\usepackage[latin1]{inputenc}

\usepackage[T1]{fontenc}        % T1 fonts
\usepackage{lmodern}            % fonts
%\usepackage{times}              % PS fonts
\usepackage{lastpage}           % to have lastpage in headers
\usepackage{url}                % urls
\usepackage{listings}
% geometry package
\usepackage[outer=20mm,inner=30mm,vmargin=20mm,includehead,includefoot,headheight=15pt]{geometry}


%% space between columns
\columnsep 10mm

% avoid widows and orphans
\clubpenalty=300
\widowpenalty=300

\usepackage[pdftex]{hyperref}
\hypersetup{%
    a4paper = true,              % use A4 paper 
    bookmarks = true,            % make bookmarks 
    colorlinks = true,           % false: boxed links; true: colored links
    pdffitwindow = false,        % page fit to window when opened
    pdfpagemode = UseNone,       % do not show bookmarks
    pdfpagelayout = SinglePage,  % displays a single page
    pdfpagetransition = Replace, % page transition
    linkcolor=blue,              % hyperlink colors
    urlcolor=blue,
    citecolor=blue,
    anchorcolor=green
}

\usepackage{indentfirst}       % indent also 1st paragraph

\pagestyle{myheadings}         % Option to put page headers
\markboth{{\small\it Exemplo de artigo em \LaTeX}}
{{\small\it Grupo xx, \today}}

%\hyphenation{}                  % explicit hyphenation

% entities
\newcommand{\class}[1]{{\normalfont\slshape #1\/}}
\newcommand{\svg}{\class{SVG}}
\newcommand{\scada}{\class{SCADA}}
\newcommand{\scadadms}{\class{SCADA/DMS}}

\title{JogginGo! }

\author{Luís Carlos Moreira Dias\\
\small Faculdade de Engenharia\\[-0.8ex]
\small da Universidade do Porto\\[-0.8ex]
\small R.\ Dr.\ Roberto Frias, 4200-465 Porto\\[-0.8ex]
\small \texttt{ei08094@fe.up.pt}\\
\and
Luís Filipe Castanheira Gomes\\
\small Faculdade de Engenharia\\[-0.8ex]
\small da Universidade do Porto\\[-0.8ex]
\small R.\ Dr.\ Roberto Frias, 4200-465 Porto\\[-0.8ex]
\small \texttt{ei08169@fe.up.pt}
}

\date{\today}

\begin{document}

\maketitle
\thispagestyle{plain} 

\begin{abstract}
Com uma interface limpa e amiga do utilizador, o JogginGo! é uma aplicação Web que permite a gestão de todas as corridas feitas por qualquer utilizador registado. Cada corrida é um tratada como um conjunto de coordenadas GPS (\textit{Global Positioning System}) recolhidas com recurso aos sensores de um dispositivo móvel. A cada minuto, intervalo de tempo definido, é recolhida a posição em que o atleta se encontra, e o conjunto de pontos é assim representado visualmente na interface \textit{web}. É depois possível também competir contra outros utilizadores, através de contra-relógio, em percursos pré-definidos pela plataforma. Utilizadores que façam jogging no seu dia a dia podem assim aproveitar a evolução dos dispositivos móveis e dos sistemas GPS para melhorarem a sua performance, ao mesmo tempo que aproveitam a componente social para maior diversão.
\end{abstract}

\section{Introdução}\label{sec:intro}

Os sistemas GPS (\emph{Global Positioning System}) são responsáveis pela recolha, em tempo real, da posição de um determinado dispositivo no mundo. Actualmente um sistema GPS é altamente fiável, garantindo com enorme exactidão a posição, em formato de coordenadas, num mapa previamente inserido. Para além da posição, através de um GPS é possível calcular dados como velocidade média, velocidade instantânea ou até o relevo.
Por outro lado, assistimos a um evoluir sistemático da utilização de dispositivos móveis, tais como os \emph{smartphones}. A cada dia, a sua utilização é mais massificada, já que são dispositivos com alta mobilidade e performance, que permitem aceder à internet em qualquer lado, a qualquer altura. Aproveitando as suas capacidades técnicas, uma percentagem altamente elevada destes dispositivos incluem sensores GPS. 
É através deste princípio que surge o JogginGo!. Utilizando as capacidades de um dispositivo móvel Android, um \emph{jogger} inicia o seu percurso, e automaticamente a aplicação móvel recolhe, em intervalos de tempo definidos, a sua posição actual. Esta informação é guardada em formato XML, e no final do percurso existe assim um conjunto estruturado de informação que é depois sincronizada com o \emph{webservice}. Recorrendo à interface web, o jogger pode ver todos os seus percursos e respectivos tempos e outra informação como, por exemplo, a velocidade média calculada com base na distância percorrida.
Para além desta introdução, onde se caracterizou o problema abordado
por este projecto, refere-se na secção \ref{sec:estart} o
estado da arte, onde são descritos os trabalhos relacionados com a
captura de coordenadas em dispositivos móveis e a sua gestão e visualização na \textit{Web} para melhor enquadramento do leitor nesta temática. 

\section{Sensores GPS em \emph{smartphones}}\label{sec:estart}

Nos últimos tempos têm surgido diversas soluções, apresentadas por empresas e \emph{developers} que trabalham na área das aplicações móveis, mais especificamente com recurso aos sensores GPS.

Os sensores GPS são sensores de posicionamento que detectam a localização do \emph{smartphone}, cujos dados podem ser recolhidos e analisados. Para devido funcionamento requerem uma conexão a 3 satélites para recolher o posicionamento em latitude e longitude, ou 4 satélites para recolher, para além destes, a altitude.
Sendo um sensor de posicionamento, existe o factor precisão a ter em conta, e este aumenta consoante o número de satélites visíveis na posição do sector. Normalmente a precisão encontra-se num intervalo de 20-50 metros, ou em casos mais favoráveis em cerca de 10 metros.

%Principalmente num \emph{smartphone}, existem alguns factores a ter em conta. O GPS %não funciona debaixo de um telhado, utiliza uma grande quantidade de bateria, precisa de %algum tempo para fixar o sinal e a existência de edifícios pode interferir no sinal do %satélite, reduzindo a precisão. 

\subsection{\emph{Tracking} em desportos}

A mobilidade de um \emph{smartphone}, aliada ao constante movimento de um atleta de \emph{jogging}, faz com que o sector dos desportos em exteriores seja consideravelmente explorado por empresas e desenvolvedores de aplicações móveis. 

A evolução da \emph{web} e a transformação de aplicações \emph{desktop} em aplicações \emph{Web 2.0} criam o cenário ideal para a troca de dados entre aplicações móveis e \emph{webservices}.

A constante necessidade de recorrer às redes sociais \emph{online}, partilhando informações e dados de uma forma simples e interactiva, faz também com que a transformação de actividades desportivas em actividades sociais esteja em constante crescendo e exploração.
 
Encontramos assim aplicações que disponibilizam o \emph{tracking} de um atleta durante o desporto e permitem a partilha em redes sociais para que este possa partilhar a sua performance com amigos e, de forma indirecta, comparar com a prestação de outros nas suas actividades.

\section{A Aplicação}\label{sec:application}

A grande aposta desta aplicação é a vertente social. O JogginGo! não só permite a gestão dos troços individuais de cada atleta, como permite uma competição, ao estilo contra-relógio, entre atletas, em percursos pré-definidos pela plataforma.
Esta aplicação assenta em duas vertentes base. A aplicação \textit{Android} e o \textit{Webservice}.
\begin{itemize}
\item \textbf{Android} --- É através da aplicação móvel que é recolhida toda a informação das coordenadas GPS que o \textit{jogger} percorre. Aquando do início da corrida, o atleta abre a aplicação e começa a recolha. A partir desse momento, a cada minuto, que é o tempo definido por defeito, é recolhida a coordenada GPS exacta de onde o atleta se encontra. Quando finalizada a corrida, o atleta deve passar a informação ao dispositivo móvel que pretende finalizar, e aí é dado por terminado o percurso. Após esta recolha de informação, o utilizador, recorrendo a uma ligação à internet, seja 3G ou Wifi, sincroniza o dispositivo com o \textit{webservice}, e toda a informação do percurso (ou percursos), que ainda não tenham sido sincronizados, são transferidos para a plataforma online, precedendo à sua respectiva representação.
\item \textbf{Webservice} --- Pode afirmar-se que é o cerne de toda a aplicação. É nele que é feita a gestão de todos os percursos dos atletas. É na interface \textit{web} que o utilizador pode visualizar todos os seus percursos, assim como o respectivo tempo. Podemos ver o dia em que foi realizada, a hora, a localidade e o país.
\end{itemize}

\subsection{Test your Limits!}
A grande inovação da plataforma JogginGo! é a inclusão da modalidade \textit{"Test your Limits"}. Nesta funcionalidade, os \textit{joggers} são convidados a competir entre si em percursos estipulados pela plataforma. Assim, cada \textit{jogger} terá a possibilidade de se transformar no campeão da sua zona, ou tentar bater o tempo de outros campeões noutros percursos. Sempre num estilo amigável, esta modalidade pretende incentivar a competição saudável, promovendo o desporto e a saúde.

\subsection {Dados XML}
Toda a informação relativa aos percursos, é armazenada numa base de dados. No entanto, é necessário ter um documento de suporte para transferir os ficheiros da aplicação móvel para o \textit{Webservice} e vice-versa. Como tal, recorremos a documentos XML para transferir a informação de uma forma coerente e estruturada. Segue uma pequena demonstração de trechos dos nosso documentos XML, usados na transferência de informação:

\lstset{
  language=XML,
  morekeywords={user,email,name,username,tracks, track,city,country,name, approved, created, id, private, updated, user,points,point,address,latitude,longitude}
}
\begin{lstlisting}
<user>
 <email>matt@joggingo.pt</email>
 <id type="integer">1</id>
 <name>Matt Damon</name>
 <username>mdamon</username>
 <track>
  <city>Porto</city>
  <country>Portugal</country>
  <name>Trilho</name>
  <user-id type="integer">1</user-id>
  <points type="array">
   <point>
    <address>
    Cais Cavaco 2, 4440-452 Vila Nova de 
    Gaia, Portugal
    </address>
    <id type="integer">2</id>
    <latitude type="float">
    41.142982
    </latitude>
    <longitude type="float">
    -8.631209
     </longitude>
   </point>
   <point>
    <address>
    Cais Capelo Ivens 17, 4440-452 
    Vila Nova de Gaia, Portugal
    </address>
    <id type="integer">3</id>
    <latitude type="float">
    41.142333
    </latitude>
    <longitude type="float">
    -8.626263
    </longitude>
   </point>address
   <point>
    <address>
    Cais de Gaia, 4440-452 
    Vila Nova de Gaia, Portugal
    </address>
    <id type="integer">4</id>
    <latitude type="float">
    41.138842
    </latitude>
    <longitude type="float">
    -8.621242
    </longitude>
   </point>
  </points>
 </track>
</user>
\end{lstlisting}

Explicando melhor, este documento pretende ser um pequeno exemplo de uma representação de um percurso feito por um utilizador, neste caso, o utilizador \textit{Matt Damon}. Como facilmente conseguimos perceber, esse \textit{user} tem um \textit{email}, um \textit{id} um nome e um \textit{username} associado. Após estas informações, segue a \textit{tag} <\textbf{track}>. A partir daqui, são passadas todas as informações necessárias para que o \textit{webservice} identifique da melhor maneira o trilho pretendido. Como podemos ver, cada percurso tem uma cidade, um país, um \textit{id}, um nome e um \textit{array} de pontos. Cada um desses pontos é caracterizado pela morada, pelo seu id, e pelas coordenadas GPS já referidas anteriomente (latitude e longitude). A morada fica guardada pois a API utilizada no googlemaps, necessita de uma morada para determinar o ponto em questão. Com esta estruturação, facilmente transferimos a informação entre os dois terminais, de uma forma simples, estruturada e de fácil leitura.


\section{Conclusões}\label{sec:conclusions}

Neste artigo abordou-se o desenvolvimento de uma aplicação que permita a monitorização de um \textit{jogger} ao longo de um percurso. Aliado à sua vertente social, esta plataforma permite fazer a ponte entre a individualidade de fazer \textit{jogging} com a possibilidade de competir saudavelmente contra outros atletas.
Posto toda a problemática, resolvemos dividir o desenvolvimento em três fases distintas:

\begin{itemize}
\item \textbf{Recolha de requisitos} -- Nesta etapa foi feito um levantamento de todos os requisitos necessários para a correcta implementação desta aplicação. Foi feito um estudo de todas as tecnologias existentes a fim de escolher aquela que melhor preenchesse o requisitos da aplicação.
\item \textbf{Recolha dos dados} -- Apesar de já existir o modelo do documento XML que vai vigorar na aplicação, ainda não foi feita a recolha da informação. Isto deve-se ao facto de o JogginGo! não utilizar nenhum dataset existente, mas sim, usar a sua aplicação móvel para a recolha dos próprios elementos. Assim, essa recolha será feita aquando da finalização do desenvolvimento  da aplicação \textit{Android}.
\item \textbf{Análise dos dados e desenvolvimento da plataforma}  -- Nesta terceira etapa pretende-se fazer a análise dos documentos XML recebidos em ambos os terminais, e desenvolver a plataforma para que possa ser usada pelo utilizador final.

No fim destas três etapas, pretendemos ter uma plataforma pronta a usar em qualquer parte do Mundo, usando documentos XML como base para a troca de informação entre 	\textit{webservice} e os dispositivos móveis.
\end{itemize}

\nocite{*}
%% auto bibliographic list 
\renewcommand{\bibname}{Referências}
% uses bibtex file
%\bibliographystyle{alpha-pt}
%\bibliographystyle{alpha}
\bibliographystyle{unsrt-pt}
%\bibliographystyle{unsrt}
\bibliography{artigo}{}

\end{document}


