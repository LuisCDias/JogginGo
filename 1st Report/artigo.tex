\documentclass[twocolumn,twoside,10pt,a4paper]{article}

\usepackage[portuguese]{babel}  % portuguese
\usepackage{graphicx}           % images: .png or .pdf w/ pdflatex; .eps w/ latex

%% For iso-8859-1 (latin1), comment next line and uncomment the second line
\usepackage[utf8]{inputenc}
%\usepackage[latin1]{inputenc}

\usepackage[T1]{fontenc}        % T1 fonts
\usepackage{lmodern}            % fonts
%\usepackage{times}              % PS fonts
\usepackage{lastpage}           % to have lastpage in headers
\usepackage{url}                % urls

% geometry package
\usepackage[outer=20mm,inner=30mm,vmargin=20mm,includehead,includefoot,headheight=15pt]{geometry}

%% space between columns
\columnsep 10mm

% avoid widows and orphans
\clubpenalty=300
\widowpenalty=300

\usepackage[pdftex]{hyperref}
\hypersetup{%
    a4paper = true,              % use A4 paper 
    bookmarks = true,            % make bookmarks 
    colorlinks = true,           % false: boxed links; true: colored links
    pdffitwindow = false,        % page fit to window when opened
    pdfpagemode = UseNone,       % do not show bookmarks
    pdfpagelayout = SinglePage,  % displays a single page
    pdfpagetransition = Replace, % page transition
    linkcolor=blue,              % hyperlink colors
    urlcolor=blue,
    citecolor=blue,
    anchorcolor=green
}

\usepackage{indentfirst}       % indent also 1st paragraph

\pagestyle{myheadings}         % Option to put page headers
\markboth{{\small\it Exemplo de artigo em \LaTeX}}
{{\small\it Grupo xx, \today}}

%\hyphenation{}                  % explicit hyphenation

% entities
\newcommand{\class}[1]{{\normalfont\slshape #1\/}}
\newcommand{\svg}{\class{SVG}}
\newcommand{\scada}{\class{SCADA}}
\newcommand{\scadadms}{\class{SCADA/DMS}}

\title{JogginGo! }

\author{Luís Carlos Moreira Dias\\
\small Faculdade de Engenharia\\[-0.8ex]
\small da Universidade do Porto\\[-0.8ex]
\small R.\ Dr.\ Roberto Frias, 4200-465 Porto\\[-0.8ex]
\small \texttt{ei08094@fe.up.pt}\\
\and
Luís Filipe Castanheira Gomes\\
\small Faculdade de Engenharia\\[-0.8ex]
\small da Universidade do Porto\\[-0.8ex]
\small R.\ Dr.\ Roberto Frias, 4200-465 Porto\\[-0.8ex]
\small \texttt{ei08169@fe.up.pt}
}

\date{\today}

\begin{document}

\maketitle
\thispagestyle{plain} 

\begin{abstract}
Com uma interface limpa e amiga do utilizador, o JogginGo! é uma aplicação Web que permite a gestão de todas as corridas feitas por qualquer utilizador registado. Cada corrida é um tratada como um conjunto de coordenadas GPS (\textit{Global Positioning System}) recolhidas com recurso aos sensores de um dispositivo móvel. A cada minuto, intervalo de tempo definido, é recolhida a posição em que o atleta se encontra, e o conjunto de pontos é assim representado visualmente na interface \textit{web}. É depois possível também competir contra outros utilizadores, através de contra-relógio, em percursos pré-definidos pela plataforma. Utilizadores que façam jogging no seu dia a dia podem assim aproveitar a evolução dos dispositivos móveis e dos sistemas GPS para melhorarem a sua performance, ao mesmo tempo que aproveitam a componente social para maior diversão.
\end{abstract}

\section{Introdução}\label{sec:intro}

Os sistemas GPS (\emph{Global Positioning System}) são responsáveis pela recolha, em tempo real, da posição de um determinado dispositivo no mundo. Actualmente um sistema GPS é altamente fiável, garantindo com enorme exactidão a posição, em formato de coordenadas, num mapa previamente inserido. Para além da posição, através de um GPS é possível calcular dados como velocidade média, velocidade instantânea ou até o relevo.
Por outro lado, assistimos a um evoluir sistemático da utilização de dispositivos móveis, tais como os \emph{smartphones}. A cada dia, a sua utilização é mais massificada, já que são dispositivos com alta mobilidade e performance, que permitem aceder à internet em qualquer lado, a qualquer altura. Aproveitando as suas capacidades técnicas, uma percentagem altamente elevada destes dispositivos incluem sensores GPS. 
É através deste princípio que surge o JogginGo!. Utilizando as capacidades de um dispositivo móvel Android, um \emph{jogger} inicia o seu percurso, e automaticamente a aplicação móvel recolhe, em intervalos de tempo definidos, a sua posição actual. Esta informação é guardada em formato XML, e no final do percurso existe assim um conjunto estruturado de informação que é depois sincronizada com o \emph{webservice}. Recorrendo à interface web, o jogger pode ver todos os seus percursos e respectivos tempos e outra informação como, por exemplo, a velocidade média calculada com base na distância percorrida.
Para além desta introdução, onde se caracterizou o problema abordado
por este projecto, refere-se na secção \ref{sec:estart} o
estado da arte, onde são descritos os trabalhos relacionados com a
captura de coordenadas em dispositivos móveis e a sua gestão e visualização na \textit{Web} para melhor enquadramento do leitor nesta temática. 

\section{Sensores GPS em \emph{smartphones}}\label{sec:estart}

Nos últimos tempos têm surgido diversas soluções, apresentadas por empresas e \emph{developers} que trabalham na área das aplicações móveis, mais especificamente com recurso aos sensores GPS.

Os sensores GPS são sensores de posicionamento que detectam a localização do \emph{smartphone}, cujos dados podem ser recolhidos e analisados. Para devido funcionamento requerem uma conexão a 3 satélites para recolher o posicionamento em latitude e longitude, ou 4 satélites para recolher, para além destes, a altitude.
Sendo um sensor de posicionamento, existe o factor precisão a ter em conta, e este aumenta consoante o número de satélites visíveis na posição do sector. Normalmente a precisão encontra-se num intervalo de 20-50 metros, ou em casos mais favoráveis em cerca de 10 metros.

%Principalmente num \emph{smartphone}, existem alguns factores a ter em conta. O GPS %não funciona debaixo de um telhado, utiliza uma grande quantidade de bateria, precisa de %algum tempo para fixar o sinal e a existência de edifícios pode interferir no sinal do %satélite, reduzindo a precisão. 

\subsection{Batik \svg{} Toolkit} \label{batik} 

Batik é um conjunto de bibliotecas baseadas em \textit{Java} que
permitem o uso de imagens \svg{} (visualização, geração ou
manipulação) em aplicações ou \textit{applets} \cite{kn:batikarchitecture}.  
O projecto Batik destina-se a fornecer ao programador alguns módulos
que permitem desenvolver soluções especificas usando \svg. 

Loren ipsum dolor sit amet, consectetuer adipiscing elit. 
Praesent sit amet sem. Maecenas eleifend facilisis leo. Vestibulum et
mi. Aliquam posuere, ante non tristique consectetuer, dui elit
scelerisque augue, eu vehicula nibh nisi ac est. Suspendisse elementum
sodales felis. Nullam laoreet fermentum urna. 

Duis eget diam. In est justo, tristique in, lacinia vel, feugiat eget,
quam. Pellentesque habitant morbi tristique senectus et netus et
malesuada fames ac turpis egestas. Fusce feugiat, elit ac placerat
fermentum, augue nisl ultricies eros, id fringilla enim sapien eu
felis. Vestibulum ante ipsum primis in faucibus orci luctus et
ultrices posuere cubilia Curae; Sed dolor mi, porttitor quis,
condimentum sed luctus. 

\section{Visualizador de Sinópticos}\label{sec:application}

A arquitectura do visualizador assenta sobre os seguintes conceitos base \cite{kn:zpmd}:
\begin{itemize}
\item \textbf{Componentes} --- Suspendisse auctor mattis augue \emph{push};
\item \textbf{Praesent} --- Sit amet sem maecenas eleifend facilisis leo;
\item \textbf{Pellentesque} --- Habitant morbi tristique senectus et netus.
\end{itemize}

Duis eget diam. In est justo, tristique in, lacinia vel, feugiat eget,
quam. Pellentesque habitant morbi tristique senectus et netus et
malesuada fames ac turpis egestas. Fusce feugiat, elit ac placerat
fermentum, augue nisl ultricies eros, id fringilla enim sapien eu
felis. Vestibulum ante ipsum primis in faucibus orci luctus et
ultrices posuere cubilia Curae; Sed dolor mi, porttitor quis,
condimentum sed luctus. 

Apresenta-se de seguida um exemplo de equação, completamente fora do contexto:
\begin{eqnarray}
CIF_1: \hspace*{5mm}F_0^j(a) &=& \frac{1}{2\pi \iota} \oint_{\gamma} \frac{F_0^j(z)}{z - a} dz\\
CIF_2: \hspace*{5mm}F_1^j(a) &=& \frac{1}{2\pi \iota} \oint_{\gamma} \frac{F_0^j(x)}{x - a} dx \label{eq:cif}
\end{eqnarray}

Na Equação~\ref{eq:cif} lorem ipsum dolor sit amet, consectetuer
adipiscing elit. Suspendisse tincidunt viverra elit. Donec tempus
vulputate mauris. Donec arcu. Vestibulum condimentum porta
justo. Curabitur ornare tincidunt lacus. Curabitur ac massa vel ante
tincidunt placerat. Cras vehicula semper elit. Curabitur gravida, est
a elementum suscipit, est eros ullamcorper quam, sed cursus velit
velit tempor neque. Duis tempor condimentum ante.

\subsection{Exemplo de Figura}

É apresentado na Figura~\ref{fig:arch} %da página~\pageref{fig:arch}
um exemplo de figura flutuante que ficará onde o \LaTeX\ entender.

\begin{figure}
  \begin{center}
    \leavevmode
    \includegraphics[width=0.45\textwidth]{puzzle}
    \caption{Arquitectura da Solução Proposta}
    \label{fig:arch}
  \end{center}
\end{figure}

Loren ipsum dolor sit amet, consectetuer adipiscing elit. 
Praesent sit amet sem. Maecenas eleifend facilisis leo. Vestibulum et
mi. Aliquam posuere, ante non tristique consectetuer, dui elit
scelerisque augue, eu vehicula nibh nisi ac est. Suspendisse elementum
sodales felis. Nullam laoreet fermentum urna. 

Duis eget diam. In est justo, tristique in, lacinia vel, feugiat eget,
quam. Pellentesque habitant morbi tristique senectus et netus et
malesuada fames ac turpis egestas. Fusce feugiat, elit ac placerat
fermentum, augue nisl ultricies eros, id fringilla enim sapien eu
felis. Vestibulum ante ipsum primis in faucibus orci luctus et
ultrices posuere cubilia Curae; Sed dolor mi, porttitor quis,
condimentum sed luctus. 

Duis eget diam. In est justo, tristique in, lacinia vel, feugiat eget,
quam. Pellentesque habitant morbi tristique senectus et netus et
malesuada fames ac turpis egestas. Fusce feugiat, elit ac placerat
fermentum, augue nisl ultricies eros, id fringilla enim sapien eu
felis. Vestibulum ante ipsum primis in faucibus orci luctus et
ultrices posuere cubilia Curae; Sed dolor mi, porttitor quis,
condimentum sed luctus. 

\subsection{Exemplo de Tabela}

É apresentado na Tabela~\ref{tab:exemplo1} um exemplo de tabela.

\begin{table}[h]
  \centering
  \caption{Uma Tabela Simples}
\begin{tabular}{| l | p{45mm} |}
	\hline
\textbf{Acrónimo} & \textbf{Significado}\\
	\hline
	\hline
        ADT   & \emph{Abstract Data Type}\\\hline
        ANDF  & \emph{Architecture-Neutral Distribution Format}\\\hline
        API   & \emph{Application Programming Interface}\\
	\hline
\end{tabular}
  \label{tab:exemplo1}
\end{table}

Duis eget diam. In est justo, tristique in, lacinia vel, feugiat eget,
quam. Pellentesque habitant morbi tristique senectus et netus et
malesuada fames ac turpis egestas. Fusce feugiat, elit ac placerat
fermentum, augue nisl ultricies eros, id fringilla enim sapien eu
felis. Vestibulum ante ipsum primis in faucibus orci luctus et
ultrices posuere cubilia Curae; Sed dolor mi, porttitor quis,
condimentum sed luctus. 

Loren ipsum dolor sit amet, consectetuer adipiscing elit. 
Praesent sit amet sem. Maecenas eleifend facilisis leo. Vestibulum et
mi. Aliquam posuere, ante non tristique consectetuer, dui elit
scelerisque augue, eu vehicula nibh nisi ac est. Suspendisse elementum
sodales felis. Nullam laoreet fermentum urna. 

Duis eget diam. In est justo, tristique in, lacinia vel, feugiat eget,
quam. Pellentesque habitant morbi tristique senectus et netus et
malesuada fames ac turpis egestas. Fusce feugiat, elit ac placerat
fermentum, augue nisl ultricies eros, id fringilla enim sapien eu
felis. Vestibulum ante ipsum primis in faucibus orci luctus et
ultrices posuere cubilia Curae; Sed dolor mi, porttitor quis,
condimentum sed luctus. 

% \begin{table}
%   \centering
%   \caption{Tabela Exemplo}
% \begin{tabular}{|c|r@{.}lr@{.}lr@{.}l||r|}
% 	\hline
% \multicolumn{8}{|c|}
% 	{\rule[-3mm]{0mm}{8mm}Iteração $k$ de $f(x_n)$} \\
% \textbf{\em k}
% 	& \multicolumn{2}{c}{$x_1^k$}
% 	& \multicolumn{2}{c}{$x_2^k$}
% 	& \multicolumn{2}{c||}{$x_3^k$}
% 	& comentários \\ \hline \hline
% 0   & -0&3                 & 0&6                 &  0&7   & - \\
% 1   &  0&47102965 & 0&04883157 & -0&53345964  & $\delta<\epsilon$ \\
% 2   &  0&49988691 & 0&00228830 & -0&52246185  & $\delta < \varepsilon$ \\
% 3   &  0&49999976 & 0&00005380 & -0&523656   &   $N$ \\
% 4   &  0&5                 & 0&00000307 & -0&52359743  & \\
% \vdots	& \multicolumn{2}{c}{\vdots}
% 	& \multicolumn{2}{c}{$\ddots$}
% 	& \multicolumn{2}{c||}{\vdots}  & \\
% 7   &  0&5   & 0&0    & \textbf{-0}&\textbf{52359878}
% 		 & $\delta<10^{-8}$ \\ \hline
% \end{tabular}
%   \label{tab:exemplo2}
% \end{table}

% Loren ipsum dolor sit amet, consectetuer adipiscing elit. 
% Praesent sit amet sem. Maecenas eleifend facilisis leo. Vestibulum et
% mi. Aliquam posuere, ante non tristique consectetuer, dui elit
% scelerisque augue, eu vehicula nibh nisi ac est. Suspendisse elementum
% sodales felis. Nullam laoreet fermentum urna. 

Duis eget diam. In est justo, tristique in, lacinia vel, feugiat eget,
quam. Pellentesque habitant morbi tristique senectus et netus et
malesuada fames ac turpis egestas. Fusce feugiat, elit ac placerat
fermentum, augue nisl ultricies eros, id fringilla enim sapien eu
felis. Vestibulum ante ipsum primis in faucibus orci luctus et
ultrices posuere cubilia Curae; Sed dolor mi, porttitor quis,
condimentum sed luctus. 

\section{Conclusões}\label{sec:conclusions}

Neste artigo abordou-se o desenvolvimento de um protótipo, com vista a
estudar a adequadibilidade da tecnologia \svg{} à visualização de
sinópticos na \textit{Web}.

Loren ipsum dolor sit amet, consectetuer adipiscing elit. 
Praesent sit amet sem. Maecenas eleifend facilisis leo. Vestibulum et
mi. Aliquam posuere, ante non tristique consectetuer, dui elit
scelerisque augue, eu vehicula nibh nisi ac est. Suspendisse elementum
sodales felis. Nullam laoreet fermentum urna. 

Duis eget diam. In est justo, tristique in, lacinia vel, feugiat eget,
quam. Pellentesque habitant morbi tristique senectus et netus et
malesuada fames ac turpis egestas. Fusce feugiat, elit ac placerat
fermentum, augue nisl ultricies eros, id fringilla enim sapien eu
felis. Vestibulum ante ipsum primis in faucibus orci luctus et
ultrices posuere cubilia Curae; Sed dolor mi, porttitor quis,
condimentum sed luctus. 

%% auto bibliographic list 
\renewcommand{\bibname}{Referências}
% uses bibtex file
%\bibliographystyle{alpha-pt}
%\bibliographystyle{alpha}
\bibliographystyle{unsrt-pt}
%\bibliographystyle{unsrt}
\bibliography{artigo}

\end{document}


