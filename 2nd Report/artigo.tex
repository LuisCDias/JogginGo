\documentclass[twocolumn,twoside,10pt,a4paper]{article}

\usepackage[portuguese]{babel}  % portuguese
\usepackage{graphicx}           % images: .png or .pdf w/ pdflatex; .eps w/ latex

%% For iso-8859-1 (latin1), comment next line and uncomment the second line
\usepackage[utf8]{inputenc}
%\usepackage[latin1]{inputenc}

\usepackage[T1]{fontenc}        % T1 fonts
\usepackage{lmodern}            % fonts
%\usepackage{times}              % PS fonts
\usepackage{lastpage}           % to have lastpage in headers
\usepackage{url}                % urls
\usepackage{listings}
% geometry package
\usepackage[outer=20mm,inner=30mm,vmargin=20mm,includehead,includefoot,headheight=15pt]{geometry}


%% space between columns
\columnsep 10mm

% avoid widows and orphans
\clubpenalty=300
\widowpenalty=300

\usepackage[pdftex]{hyperref}
\hypersetup{%
    a4paper = true,              % use A4 paper 
    bookmarks = true,            % make bookmarks 
    colorlinks = true,           % false: boxed links; true: colored links
    pdffitwindow = false,        % page fit to window when opened
    pdfpagemode = UseNone,       % do not show bookmarks
    pdfpagelayout = SinglePage,  % displays a single page
    pdfpagetransition = Replace, % page transition
    linkcolor=blue,              % hyperlink colors
    urlcolor=blue,
    citecolor=blue,
    anchorcolor=green
}

\usepackage{indentfirst}       % indent also 1st paragraph

\pagestyle{myheadings}         % Option to put page headers
\markboth{{\small\it Exemplo de artigo em \LaTeX}}
{{\small\it Grupo 10, \today}}

%\hyphenation{}                  % explicit hyphenation

% entities
\newcommand{\class}[1]{{\normalfont\slshape #1\/}}
\newcommand{\svg}{\class{SVG}}
\newcommand{\scada}{\class{SCADA}}
\newcommand{\scadadms}{\class{SCADA/DMS}}

\title{JogginGo! }

\author{Luís Carlos Moreira Dias\\
\small Faculdade de Engenharia\\[-0.8ex]
\small da Universidade do Porto\\[-0.8ex]
\small R.\ Dr.\ Roberto Frias, 4200-465 Porto\\[-0.8ex]
\small \texttt{ei08094@fe.up.pt}\\
\and
Luís Filipe Castanheira Gomes\\
\small Faculdade de Engenharia\\[-0.8ex]
\small da Universidade do Porto\\[-0.8ex]
\small R.\ Dr.\ Roberto Frias, 4200-465 Porto\\[-0.8ex]
\small \texttt{ei08169@fe.up.pt}
}

\date{\today}

\begin{document}

\maketitle
\thispagestyle{plain} 

\begin{abstract}
Com uma interface limpa e amiga do utilizador, o JogginGo! é uma aplicação Web que permite a gestão de todas as corridas feitas por qualquer utilizador registado. Cada corrida é tratada como um conjunto de coordenadas GPS (\textit{Global Positioning System}) recolhidas com recurso aos sensores de um dispositivo móvel. A cada minuto, intervalo de tempo definido, é recolhida a posição em que o atleta se encontra, e o conjunto de pontos é assim representado visualmente na interface \textit{web}. É depois possível também competir contra outros utilizadores, através de contra-relógio, em percursos pré-definidos pela plataforma. Utilizadores que façam jogging no seu dia a dia podem assim aproveitar a evolução dos dispositivos móveis e dos sistemas GPS para melhorarem a sua performance, ao mesmo tempo que aproveitam a componente social para maior diversão.
\end{abstract}

\section{Introdução}\label{sec:intro}

Os sistemas GPS (\emph{Global Positioning System}) são responsáveis pela recolha, em tempo real, da posição de um determinado dispositivo no mundo. Actualmente um sistema GPS é altamente fiável, garantindo com enorme exactidão a posição, em formato de coordenadas, num mapa previamente inserido. Para além da posição, através de um GPS é possível calcular dados como velocidade média, velocidade instantânea ou até o relevo.
Por outro lado, assistimos a um evoluir sistemático da utilização de dispositivos móveis, tais como os \emph{smartphones}. A cada dia, a sua utilização é mais massificada, já que são dispositivos com alta mobilidade e performance, que permitem aceder à internet em qualquer lado, a qualquer altura. Aproveitando as suas capacidades técnicas, uma percentagem altamente elevada destes dispositivos incluem sensores GPS. 
É através deste princípio que surge o \textit{"JogginGo!"}. Utilizando as capacidades de um dispositivo móvel Android, um \emph{jogger} inicia o seu percurso, e automaticamente a aplicação móvel recolhe, em intervalos de tempo definidos, a sua posição actual. Esta informação é guardada em formato XML, e no final do percurso existe assim um conjunto estruturado de informação que é depois sincronizada com o \emph{webservice}. Recorrendo à interface web, o jogger pode ver todos os seus percursos e respectivos tempos e outra informação como, por exemplo, a velocidade média calculada com base na distância percorrida.
Para além desta introdução, onde se caracterizou o problema abordado
por este projecto, refere-se na secção \ref{sec:estart} o
estado da arte, onde são descritos os trabalhos relacionados com a
captura de coordenadas em dispositivos móveis e a sua gestão e visualização na \textit{Web} para melhor enquadramento do leitor nesta temática. 

\section{Estado da Arte}\label{sec:estarte}

Hoje em dia existem diversas soluções, apresentadas por empresas e \emph{developers} que trabalham na área das aplicações móveis, que fazem o \textit{tracking} da posição de um \emph{smartphone} com recurso aos sensores GPS, isto é, sensores de posicionamento que detectam a localização do \emph{smartphone} e cujos dados podem ser recolhidos e analisados posteriormente.

\subsection{\emph{Tracking} em desportos}

A mobilidade de um \emph{smartphone}, aliada ao constante movimento de um atleta de \emph{jogging}, faz com que o sector dos desportos em exteriores seja consideravelmente explorado.

A evolução da \emph{web} e a transformação de aplicações \emph{desktop} em aplicações \emph{Web 2.0} criam o cenário ideal para a troca de dados entre aplicações móveis e \emph{webservices}.

A constante necessidade de recorrer às redes sociais \emph{online}, partilhando informações e dados de uma forma simples e interactiva, faz também com que a transformação de actividades desportivas em actividades sociais esteja em constante crescendo e exploração.
 
Encontramos assim aplicações que disponibilizam o \emph{tracking} de um atleta durante o desporto e permitem a partilha em redes sociais para que este possa partilhar a sua performance com amigos e, de forma indirecta, comparar com a prestação de outros nas suas actividades.

\subsubsection{\emph{Sports Tracker}}

Esta aplicação, da empresa \textit{Sports Tracking Technologies Ltd.}, permite, citando a sua página, "Transformar o seu \textit{smartphone} num computador de desporto social". Permite fazer o \textit{tracking} de um atleta em vários desportos, não sendo específico para o \textit{jogging}. 

Tem como principais funcionalidades a análise da \textit{performance}, seguir o progresso de um atleta ao longo dos treinos, guardar os dados dos treinos, a utilização de mapas para melhor visualização por parte do utilizador,  e, numa componente mais social, partilhar informação e fotografias sobre os treinos, assim como visualizar estas mesmas partilhas por parte de outros utilizadores da plataforma.

\subsubsection{\emph{Endomondo Sports Tracker}}

Desenvolvida pela \textit{Endomondo}, permite também fazer o \textit{tracking} de um atleta em vários desportos, não sendo específico para a corrida, mas também para, por exemplo, ciclismo e caminhada.

Tem como principais características a possibilidade de escolher de entre 3 programas de treino, ou criar um personalizado, e ter \textit{feedback} de um "treinador áudio", a possibilidade de se superar a si próprio, estabelecendo um treino prévio como objectivo, a visualização de gráficos com tempos parciais, frequência cardíaca, velocidade e altitude ao longo dos exercícios, a partilha informações do treino nas redes sociais e, na versão mais recente da aplicação, correr contra o tempo de um amigo ou até competir numa rota específica e tentar superar o "campeão" dela.

\section{Requisitos do Utilizador}\label{sec:application}

Para o uso da plataforma, qualquer utilizador necessita de ter um email válido, para que se possa registar. Após o registo estar concluído, o utilizador tem acesso a uma série de funcionalidades. Nesta arquitectura do \textit{JogginGo!} existem três níveis de utilizadores.

\begin{itemize}
\item {Utilizador} -- Este tipo de utilizador é o utilizador final. Ou seja, o cliente alvo para o \textit{JogginGo!}. A este tipo de utilizador é permitido fazer as suas próprias corridas, competir em corridas pré-definidas e sugerir que alguma(s) da(s) sua(s) corrida(s) passem a pré-definidas, pertencentes à funcionalidade \textit{Test your limits}.
\item{Moderador} -- Os moderadores têm a função de garantir o bom funcionamento da plataforma, assim como prestar auxílio aos utilizadores na utilização do sistema. Para além disso, os moderadores têm como responsabilidade a aprovação, ou não, do pedido de um utilizador a que a sua corrida seja tida em conta como pertencente à funcionalidade \textit{"Test your limits"}. Para tal, o moderador deve avaliar alguns factores, nomeadamente a dificuldade do percurso ou a área involvente (se há ou não alguma corrida pré-definida nas proximidades, se a paisagem é agradável, etc.).
\item{Administrador} -- O Administrador existe para actuar ou resolver algum conflito que saia da área de acção do moderador. A reter, o administrador pode banir um utilizador, apagar alguma corrida, introduzir alguma corrida pré-definida, etc.
\end{itemize}

\section{Arquitectura da solução}
Tal como temos vindo a mostrar ao longo do desenvolvimento do projecto, um dos principais problemas prende-se na forma como os dados irão ser armazenados, a sua estrutura, para que sejam facilmente interpretáveis. Para responder a este problema, optámos por utilizar uma linguagem de marcação (XML) como base para a estruturação dos dados. No entanto, cada documento XML, que contém a informação de cada corrida, tem que guardar dados relativos ao utilizador, como o seu id, e dados da própria localização da corrida. De seguida, apresentamos a nossa solução para este problema.

INTRODUZIR O XML LINDO AQUI

Explicando melhor, este documento é um exemplo de uma representação de um percurso feito por um utilizador, neste caso, o utilizador com o \textit{user\_id} igual a 1. Há outros campos que descodificamos facilmente o seu significado, como \textit{city}, que representa a cidade onde foi feita a corrida, o \textit{name}, que é o nome que o utilizador deu à sua corrida, ou mesmo os campos referentes ao tempo, que tem tempo inicial e final. No entanto, viremos a nossa atenção para os campos \textit{approved} e \textit{private}, com o valor \textit{false} e \textit{true} respectivamente. Por defeito, qualquer corrida criada por um utilizador tem o valor tem esses valores para os campos referidos anteriormente. Isto foi uma implementação para diferenciar os trilhos que são privados, os que são privados à espera de aprovação para serem públicos, e os públicos. Ou seja, quando um user cria um trilho, \textit{approved} e \textit{private} tomam os valores \textit{false} e \textit{true}, respectivamente. Após isso, o utilizador pode pretender que o seu trilho seja passado para a funcionalidade \textit{Test you limits}. Assim, \textit{private} passa a valer \textit{true}. No entanto, \textit{approved} continua com o valor booleano para falso. Isto porque esse trilho ainda necessida de aprovação de um moderador para que passe a público. Se for aprovado, \textit{approved} passa a \textit{true} e o trilho passa a ser público. Caso contrário \textit{approved} mantém o valor \textit{false}, e \textit{private} é alterado para \textit{true}, exactamente igual ao momento da sua criação.
Outra particularidade deste documento XML é a existência de um um \textit{array} de pontos. Cada posição do \textit{array} é um objecto JSON que contém a latitude e a longitude de cada ponto recolhido pelo dispositivo móvel.
Este documento XML visa resolver o problema de estruturação da informação a ser trocada entre o dispositivo móvel e o \textit{webservice}. 

\section{Trabalho desenvolvido}\label{sec:trabalhodesenvolvido}

\subsection{Aplicação Android}

Até este momento, dada a preparação do  \textit{webservice} para receber e tratar os dados, a aplicação Android encontra-se numa fase algo rudimentar. No entanto, já está desenvolvido um protótipo que recolhe as coordenadas  \textit{GPS} e as apresenta no ecrã, que permite também visualizar um mapa recorrendo à \emph{API Google Maps}, ainda sem ligação às coordenadas obtidas.
O trabalho a continuar passa pela automatização da recolha de coordenadas ao longo de intervalos de tempo, apresentar o mapa centrado nestas mesmas coordenadas e desenhar o percurso até lá, e guardar toda esta informação em formato XML para que possa depois ser analizado e sincronizado com o  \textit{webservice}. Para isto, terá também de ser desenvolvida a componente de autenticação da aplicação com o  \textit{webservice}.
Depois deste trabalho, é intenção avançar para a componente social da aplicação, podendo partilhar a partir do  \textit{smartphone} as informações nas redes sociais.


\section{Conclusões}\label{sec:conclusions}

Neste artigo abordou-se o desenvolvimento de uma aplicação que permita a monitorização de um \textit{jogger} ao longo de um percurso. Aliado à sua vertente social, esta plataforma permite fazer a ponte entre a individualidade de fazer \textit{jogging} com a possibilidade de competir saudavelmente contra outros atletas.
Posto toda a problemática, resolvemos dividir o desenvolvimento em três fases distintas:

\begin{itemize}
\item \textbf{Recolha de requisitos} -- Nesta etapa foi feito um levantamento de todos os requisitos necessários para a correcta implementação desta aplicação. Foi feito um estudo de todas as tecnologias existentes a fim de escolher aquela que melhor preenchesse o requisitos da aplicação.
\item \textbf{Recolha dos dados} -- Apesar de já existir o modelo do documento XML que vai vigorar na aplicação, ainda não foi feita a recolha da informação. Isto deve-se ao facto de o JogginGo! não utilizar nenhum dataset existente, mas sim, usar a sua aplicação móvel para a recolha dos próprios elementos. Assim, essa recolha será feita aquando da finalização do desenvolvimento  da aplicação \textit{Android}.
\item \textbf{Análise dos dados e desenvolvimento da plataforma}  -- Nesta terceira etapa pretende-se fazer a análise dos documentos XML recebidos em ambos os terminais, e desenvolver a plataforma para que possa ser usada pelo utilizador final.
\end{itemize}

No fim destas três etapas, pretendemos ter uma plataforma pronta a usar em qualquer parte do Mundo, usando documentos XML como base para a troca de informação entre 	\textit{webservice} e os dispositivos móveis.


\nocite{*}
%% auto bibliographic list 
\renewcommand{\bibname}{Referências}
% uses bibtex file
%\bibliographystyle{alpha-pt}
%\bibliographystyle{alpha}
\bibliographystyle{unsrt-pt}
%\bibliographystyle{unsrt}
\bibliography{artigo}{}

\end{document}


